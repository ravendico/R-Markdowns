\documentclass[11pt - \usepackage{enumerate},]{article}
\usepackage{lmodern}
\usepackage{amssymb,amsmath}
\usepackage{ifxetex,ifluatex}
\usepackage{fixltx2e} % provides \textsubscript
\ifnum 0\ifxetex 1\fi\ifluatex 1\fi=0 % if pdftex
  \usepackage[T1]{fontenc}
  \usepackage[utf8]{inputenc}
\else % if luatex or xelatex
  \ifxetex
    \usepackage{mathspec}
  \else
    \usepackage{fontspec}
  \fi
  \defaultfontfeatures{Ligatures=TeX,Scale=MatchLowercase}
\fi
% use upquote if available, for straight quotes in verbatim environments
\IfFileExists{upquote.sty}{\usepackage{upquote}}{}
% use microtype if available
\IfFileExists{microtype.sty}{%
\usepackage{microtype}
\UseMicrotypeSet[protrusion]{basicmath} % disable protrusion for tt fonts
}{}
\usepackage[margin=1in]{geometry}
\usepackage{hyperref}
\hypersetup{unicode=true,
            pdftitle={Statistical Modeling Course},
            pdfborder={0 0 0},
            breaklinks=true}
\urlstyle{same}  % don't use monospace font for urls
\usepackage{color}
\usepackage{fancyvrb}
\newcommand{\VerbBar}{|}
\newcommand{\VERB}{\Verb[commandchars=\\\{\}]}
\DefineVerbatimEnvironment{Highlighting}{Verbatim}{commandchars=\\\{\}}
% Add ',fontsize=\small' for more characters per line
\usepackage{framed}
\definecolor{shadecolor}{RGB}{248,248,248}
\newenvironment{Shaded}{\begin{snugshade}}{\end{snugshade}}
\newcommand{\KeywordTok}[1]{\textcolor[rgb]{0.13,0.29,0.53}{\textbf{#1}}}
\newcommand{\DataTypeTok}[1]{\textcolor[rgb]{0.13,0.29,0.53}{#1}}
\newcommand{\DecValTok}[1]{\textcolor[rgb]{0.00,0.00,0.81}{#1}}
\newcommand{\BaseNTok}[1]{\textcolor[rgb]{0.00,0.00,0.81}{#1}}
\newcommand{\FloatTok}[1]{\textcolor[rgb]{0.00,0.00,0.81}{#1}}
\newcommand{\ConstantTok}[1]{\textcolor[rgb]{0.00,0.00,0.00}{#1}}
\newcommand{\CharTok}[1]{\textcolor[rgb]{0.31,0.60,0.02}{#1}}
\newcommand{\SpecialCharTok}[1]{\textcolor[rgb]{0.00,0.00,0.00}{#1}}
\newcommand{\StringTok}[1]{\textcolor[rgb]{0.31,0.60,0.02}{#1}}
\newcommand{\VerbatimStringTok}[1]{\textcolor[rgb]{0.31,0.60,0.02}{#1}}
\newcommand{\SpecialStringTok}[1]{\textcolor[rgb]{0.31,0.60,0.02}{#1}}
\newcommand{\ImportTok}[1]{#1}
\newcommand{\CommentTok}[1]{\textcolor[rgb]{0.56,0.35,0.01}{\textit{#1}}}
\newcommand{\DocumentationTok}[1]{\textcolor[rgb]{0.56,0.35,0.01}{\textbf{\textit{#1}}}}
\newcommand{\AnnotationTok}[1]{\textcolor[rgb]{0.56,0.35,0.01}{\textbf{\textit{#1}}}}
\newcommand{\CommentVarTok}[1]{\textcolor[rgb]{0.56,0.35,0.01}{\textbf{\textit{#1}}}}
\newcommand{\OtherTok}[1]{\textcolor[rgb]{0.56,0.35,0.01}{#1}}
\newcommand{\FunctionTok}[1]{\textcolor[rgb]{0.00,0.00,0.00}{#1}}
\newcommand{\VariableTok}[1]{\textcolor[rgb]{0.00,0.00,0.00}{#1}}
\newcommand{\ControlFlowTok}[1]{\textcolor[rgb]{0.13,0.29,0.53}{\textbf{#1}}}
\newcommand{\OperatorTok}[1]{\textcolor[rgb]{0.81,0.36,0.00}{\textbf{#1}}}
\newcommand{\BuiltInTok}[1]{#1}
\newcommand{\ExtensionTok}[1]{#1}
\newcommand{\PreprocessorTok}[1]{\textcolor[rgb]{0.56,0.35,0.01}{\textit{#1}}}
\newcommand{\AttributeTok}[1]{\textcolor[rgb]{0.77,0.63,0.00}{#1}}
\newcommand{\RegionMarkerTok}[1]{#1}
\newcommand{\InformationTok}[1]{\textcolor[rgb]{0.56,0.35,0.01}{\textbf{\textit{#1}}}}
\newcommand{\WarningTok}[1]{\textcolor[rgb]{0.56,0.35,0.01}{\textbf{\textit{#1}}}}
\newcommand{\AlertTok}[1]{\textcolor[rgb]{0.94,0.16,0.16}{#1}}
\newcommand{\ErrorTok}[1]{\textcolor[rgb]{0.64,0.00,0.00}{\textbf{#1}}}
\newcommand{\NormalTok}[1]{#1}
\usepackage{graphicx,grffile}
\makeatletter
\def\maxwidth{\ifdim\Gin@nat@width>\linewidth\linewidth\else\Gin@nat@width\fi}
\def\maxheight{\ifdim\Gin@nat@height>\textheight\textheight\else\Gin@nat@height\fi}
\makeatother
% Scale images if necessary, so that they will not overflow the page
% margins by default, and it is still possible to overwrite the defaults
% using explicit options in \includegraphics[width, height, ...]{}
\setkeys{Gin}{width=\maxwidth,height=\maxheight,keepaspectratio}
\IfFileExists{parskip.sty}{%
\usepackage{parskip}
}{% else
\setlength{\parindent}{0pt}
\setlength{\parskip}{6pt plus 2pt minus 1pt}
}
\setlength{\emergencystretch}{3em}  % prevent overfull lines
\providecommand{\tightlist}{%
  \setlength{\itemsep}{0pt}\setlength{\parskip}{0pt}}
\setcounter{secnumdepth}{0}
% Redefines (sub)paragraphs to behave more like sections
\ifx\paragraph\undefined\else
\let\oldparagraph\paragraph
\renewcommand{\paragraph}[1]{\oldparagraph{#1}\mbox{}}
\fi
\ifx\subparagraph\undefined\else
\let\oldsubparagraph\subparagraph
\renewcommand{\subparagraph}[1]{\oldsubparagraph{#1}\mbox{}}
\fi

%%% Use protect on footnotes to avoid problems with footnotes in titles
\let\rmarkdownfootnote\footnote%
\def\footnote{\protect\rmarkdownfootnote}

%%% Change title format to be more compact
\usepackage{titling}

% Create subtitle command for use in maketitle
\newcommand{\subtitle}[1]{
  \posttitle{
    \begin{center}\large#1\end{center}
    }
}

\setlength{\droptitle}{-2em}

  \title{\textbf{Statistical Modeling Course}}
    \pretitle{\vspace{\droptitle}\centering\huge}
  \posttitle{\par}
  \subtitle{\textbf{Logistic Regression Assignment}}
  \author{}
    \preauthor{}\postauthor{}
    \date{}
    \predate{}\postdate{}
  
\usepackage{setspace}\onehalfspacing

\begin{document}
\maketitle

We will use the \texttt{Default} data set in the \texttt{ISLR} package
for this assignment.

\section{Problem 1}\label{problem-1}

Fit a logistic regression model that uses \texttt{student},
\texttt{income} and \texttt{balance} to predict \texttt{default} .
Interpret the coefficients.

\begin{Shaded}
\begin{Highlighting}[]
\KeywordTok{data}\NormalTok{(}\StringTok{"Default"}\NormalTok{)}
\NormalTok{model <-}\StringTok{ }\KeywordTok{glm}\NormalTok{(default}\OperatorTok{~}\NormalTok{student}\OperatorTok{+}\NormalTok{income}\OperatorTok{+}\NormalTok{balance,}\DataTypeTok{data=}\NormalTok{Default, }\DataTypeTok{family=}\StringTok{"binomial"}\NormalTok{)}
\KeywordTok{summary}\NormalTok{(model)}
\end{Highlighting}
\end{Shaded}

\begin{verbatim}
## 
## Call:
## glm(formula = default ~ student + income + balance, family = "binomial", 
##     data = Default)
## 
## Deviance Residuals: 
##     Min       1Q   Median       3Q      Max  
## -2.4691  -0.1418  -0.0557  -0.0203   3.7383  
## 
## Coefficients:
##               Estimate Std. Error z value Pr(>|z|)    
## (Intercept) -1.087e+01  4.923e-01 -22.080  < 2e-16 ***
## studentYes  -6.468e-01  2.363e-01  -2.738  0.00619 ** 
## income       3.033e-06  8.203e-06   0.370  0.71152    
## balance      5.737e-03  2.319e-04  24.738  < 2e-16 ***
## ---
## Signif. codes:  0 '***' 0.001 '**' 0.01 '*' 0.05 '.' 0.1 ' ' 1
## 
## (Dispersion parameter for binomial family taken to be 1)
## 
##     Null deviance: 2920.6  on 9999  degrees of freedom
## Residual deviance: 1571.5  on 9996  degrees of freedom
## AIC: 1579.5
## 
## Number of Fisher Scoring iterations: 8
\end{verbatim}

\begin{Shaded}
\begin{Highlighting}[]
\KeywordTok{exp}\NormalTok{(}\KeywordTok{coef}\NormalTok{(model))}
\end{Highlighting}
\end{Shaded}

\begin{verbatim}
##  (Intercept)   studentYes       income      balance 
## 1.903854e-05 5.237317e-01 1.000003e+00 1.005753e+00
\end{verbatim}

\textit{Answer: } The coefficients are as follows:

\begin{itemize}
\item (Intercept) -1.087e+01
\item student     -6.468e-01
\item income       3.033e-06
\item balance      5.737e-03
\end{itemize}

However, using a p-value of 0.05, only the intercept, student, and
balance predictor variables are statistically significant. The
coefficients are much easier to interpret by using \(e^{X^T\beta}\).The
intercept is the odds of default when each one of the predictor
variables is equal to 0. The student coefficient of -0.6468 indicates
the odds of default decreases when the person is a student for fixed
number of income and balance which is quite surprising. For income, the
odds of default increases by 3.033e-06\% as the income of a person
increases by one unit given fixed student type and balance. Similarly,
the odds of default increases by 5.737e-03\% as the average balance that
the customer has remaining on their credit card after making their
monthly payment increases by one unit.

\section{Problem 2}\label{problem-2}

Using the validation set approach, estimate the test error of this
model. To do this, perform the following steps:

\begin{itemize}
\tightlist
\item
  Write a function that takes 2 arguments: a formula and a dateset
\item
  The function should do the following:
\item
  Split the sample set intro a training set and a validation set.
\item
  Fit a multiple logistic regression model using only the training
  observations.
\item
  Obtain a prediction of default status for each individual in the
  validation set by computing the estimated probability of default for
  that individual, and classifying the individual to the
  \texttt{default} category if the estimated probability is greater than
  0.5.
\item
  Compute the validation set error, which is the fraction of the
  observations in the validation set that are misclassified.
\item
  The function should return the validation error.
\end{itemize}

\begin{Shaded}
\begin{Highlighting}[]
\KeywordTok{set.seed}\NormalTok{(}\DecValTok{2019}\NormalTok{)}
\NormalTok{prob2 <-}\StringTok{ }\ControlFlowTok{function}\NormalTok{(formula,dataset)\{}
\NormalTok{  train_index =}\StringTok{ }\KeywordTok{sample}\NormalTok{(}\KeywordTok{c}\NormalTok{(}\OtherTok{TRUE}\NormalTok{,}\OtherTok{FALSE}\NormalTok{), }\KeywordTok{nrow}\NormalTok{(dataset), }\DataTypeTok{replace=}\OtherTok{TRUE}\NormalTok{, }\DataTypeTok{prob=}\KeywordTok{c}\NormalTok{(}\FloatTok{0.6}\NormalTok{,}\FloatTok{0.4}\NormalTok{))}
\NormalTok{  train <-}\StringTok{ }\NormalTok{dataset[train_index,]}
\NormalTok{  validation <-}\StringTok{ }\NormalTok{dataset[}\OperatorTok{!}\NormalTok{train_index,]}
\NormalTok{  model <-}\StringTok{ }\KeywordTok{glm}\NormalTok{(formula, }\DataTypeTok{data=}\NormalTok{train, }\DataTypeTok{family=}\StringTok{"binomial"}\NormalTok{)}
\NormalTok{  predicted_probs =}\StringTok{ }\KeywordTok{predict}\NormalTok{(model, validation, }\DataTypeTok{type=}\StringTok{"response"}\NormalTok{)}\OperatorTok{>}\FloatTok{0.5}
\NormalTok{  predict <-}\StringTok{ }\KeywordTok{rep}\NormalTok{(}\StringTok{"No"}\NormalTok{,}\KeywordTok{nrow}\NormalTok{(validation))}
\NormalTok{  predict[predicted_probs}\OperatorTok{>}\FloatTok{0.5}\NormalTok{] =}\StringTok{ "Yes"}
  \KeywordTok{return}\NormalTok{(}\KeywordTok{mean}\NormalTok{(predict }\OperatorTok{!=}\StringTok{ }\NormalTok{validation}\OperatorTok{$}\NormalTok{default))}
\NormalTok{\}}
\end{Highlighting}
\end{Shaded}

\section{Problem 3}\label{problem-3}

Use your function from Problem 2 to repeat the process ten times, using
ten different splits of the observations into a training set and a
validation set, then get the average of these test errors. Comment on
the results obtained.

\begin{Shaded}
\begin{Highlighting}[]
\KeywordTok{set.seed}\NormalTok{(}\DecValTok{2020}\NormalTok{)}
\NormalTok{errors =}\StringTok{ }\KeywordTok{rep}\NormalTok{(}\DecValTok{0}\NormalTok{,}\DecValTok{10}\NormalTok{)}
\ControlFlowTok{for}\NormalTok{(i }\ControlFlowTok{in} \DecValTok{1}\OperatorTok{:}\DecValTok{10}\NormalTok{)\{}
\NormalTok{  errors[i]=}\KeywordTok{prob2}\NormalTok{(}\DataTypeTok{formula=}\NormalTok{default }\OperatorTok{~}\StringTok{ }\NormalTok{student }\OperatorTok{+}\StringTok{ }\NormalTok{income }\OperatorTok{+}\StringTok{ }\NormalTok{balance, Default)}
\NormalTok{\}}
\KeywordTok{mean}\NormalTok{(errors)}
\end{Highlighting}
\end{Shaded}

\begin{verbatim}
## [1] 0.02715666
\end{verbatim}

\textit{Answer:} On average, the rate of misclassification for the
Default dataset using the logistic regression model, is only 2.72\%
which is a fairly acceptable amount of error.

\section{Problem 4}\label{problem-4}

Using your choice of goodness of fit test, determine if the logistic
regression model is reliable for inference. Compare the model's
descriptive and explanatory power from its predictive accuracy in
Problem 3.

\begin{Shaded}
\begin{Highlighting}[]
\KeywordTok{anova}\NormalTok{(model, }\DataTypeTok{test =} \StringTok{"Chisq"}\NormalTok{)}
\end{Highlighting}
\end{Shaded}

\begin{verbatim}
## Analysis of Deviance Table
## 
## Model: binomial, link: logit
## 
## Response: default
## 
## Terms added sequentially (first to last)
## 
## 
##         Df Deviance Resid. Df Resid. Dev  Pr(>Chi)    
## NULL                     9999     2920.7              
## student  1    11.97      9998     2908.7 0.0005416 ***
## income   1     1.19      9997     2907.5 0.2758849    
## balance  1  1335.95      9996     1571.5 < 2.2e-16 ***
## ---
## Signif. codes:  0 '***' 0.001 '**' 0.01 '*' 0.05 '.' 0.1 ' ' 1
\end{verbatim}

\textit{Answer:} Using an ANOVA test, the summary suggests that both the
balance and student status are effective in prediting the probability of
default of an individual.

\section{Problem 5}\label{problem-5}

Add your predicted probabilites to the Default data frame and call the
column \texttt{preds}. Use the following code to plot the ROC curve.
Calculate the confusion matrix and AUC. Comment on the results results.

\begin{Shaded}
\begin{Highlighting}[]
\NormalTok{Default <-}\StringTok{ }
\StringTok{  }\NormalTok{Default }\OperatorTok
\StringTok{  }\KeywordTok{mutate}\NormalTok{(}\DataTypeTok{preds =} \KeywordTok{predict}\NormalTok{(model,Default, }\DataTypeTok{type=}\StringTok{"response"}\NormalTok{))}\OperatorTok
\StringTok{  }\KeywordTok{mutate}\NormalTok{(}\DataTypeTok{default_num =} \KeywordTok{as.numeric}\NormalTok{(default)}\OperatorTok{-}\DecValTok{1}\NormalTok{)}

\KeywordTok{ggplot}\NormalTok{(Default, }\KeywordTok{aes}\NormalTok{(}\DataTypeTok{d=}\NormalTok{default_num, }\DataTypeTok{m=}\NormalTok{preds)) }\OperatorTok{+}\StringTok{ }
\StringTok{  }\KeywordTok{geom_roc}\NormalTok{(}\DataTypeTok{n.cuts =} \DecValTok{6}\NormalTok{, }\DataTypeTok{labelround =} \DecValTok{4}\NormalTok{) }\OperatorTok{+}
\StringTok{  }\KeywordTok{geom_abline}\NormalTok{(}\DataTypeTok{intercept =} \DecValTok{0}\NormalTok{, }\DataTypeTok{slope =} \DecValTok{1}\NormalTok{)}

\CommentTok{#Computing the Area Under the Curve (AUC)}
\KeywordTok{prediction}\NormalTok{(}\KeywordTok{predict}\NormalTok{(model, Default, }\DataTypeTok{type=}\StringTok{"response"}\NormalTok{), Default}\OperatorTok{$}\NormalTok{default) }\OperatorTok
\StringTok{  }\KeywordTok{performance}\NormalTok{(}\DataTypeTok{measure =} \StringTok{"auc"}\NormalTok{) }\OperatorTok
\StringTok{  }\NormalTok{.}\OperatorTok{@}\NormalTok{y.values}


\NormalTok{Default <-}\StringTok{ }\NormalTok{Default }\OperatorTok
\StringTok{  }\KeywordTok{mutate}\NormalTok{(}\DataTypeTok{default_pred =} \KeywordTok{ifelse}\NormalTok{(preds}\OperatorTok{<}\FloatTok{0.5}\NormalTok{,}\StringTok{"No"}\NormalTok{,}\StringTok{"Yes"}\NormalTok{))}

\CommentTok{#Confusion Matrix}
\KeywordTok{table}\NormalTok{(Default}\OperatorTok{$}\NormalTok{default,Default}\OperatorTok{$}\NormalTok{default_pred)}
\end{Highlighting}
\end{Shaded}

\textit{Answer: } From the graph presented above, it looks like the
model discriminates well between customers who default and those who do
not. After computing for the Area Under the Curve (AUC), a value of
0.9496 is obtained which indicates that the model is a good classifying
model. From the confusion matrix, there are numerous misclassifications
from the predicted model. False positives are 40 and false negatives are
228.


\end{document}
