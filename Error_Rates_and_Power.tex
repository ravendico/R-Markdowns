\documentclass[]{article}
\usepackage{lmodern}
\usepackage{amssymb,amsmath}
\usepackage{ifxetex,ifluatex}
\usepackage{fixltx2e} % provides \textsubscript
\ifnum 0\ifxetex 1\fi\ifluatex 1\fi=0 % if pdftex
  \usepackage[T1]{fontenc}
  \usepackage[utf8]{inputenc}
\else % if luatex or xelatex
  \ifxetex
    \usepackage{mathspec}
  \else
    \usepackage{fontspec}
  \fi
  \defaultfontfeatures{Ligatures=TeX,Scale=MatchLowercase}
\fi
% use upquote if available, for straight quotes in verbatim environments
\IfFileExists{upquote.sty}{\usepackage{upquote}}{}
% use microtype if available
\IfFileExists{microtype.sty}{%
\usepackage{microtype}
\UseMicrotypeSet[protrusion]{basicmath} % disable protrusion for tt fonts
}{}
\usepackage[margin=1in]{geometry}
\usepackage{hyperref}
\hypersetup{unicode=true,
            pdftitle={BUDS Training: Error rates, coverage rates, and power Lab},
            pdfborder={0 0 0},
            breaklinks=true}
\urlstyle{same}  % don't use monospace font for urls
\usepackage{color}
\usepackage{fancyvrb}
\newcommand{\VerbBar}{|}
\newcommand{\VERB}{\Verb[commandchars=\\\{\}]}
\DefineVerbatimEnvironment{Highlighting}{Verbatim}{commandchars=\\\{\}}
% Add ',fontsize=\small' for more characters per line
\usepackage{framed}
\definecolor{shadecolor}{RGB}{248,248,248}
\newenvironment{Shaded}{\begin{snugshade}}{\end{snugshade}}
\newcommand{\KeywordTok}[1]{\textcolor[rgb]{0.13,0.29,0.53}{\textbf{#1}}}
\newcommand{\DataTypeTok}[1]{\textcolor[rgb]{0.13,0.29,0.53}{#1}}
\newcommand{\DecValTok}[1]{\textcolor[rgb]{0.00,0.00,0.81}{#1}}
\newcommand{\BaseNTok}[1]{\textcolor[rgb]{0.00,0.00,0.81}{#1}}
\newcommand{\FloatTok}[1]{\textcolor[rgb]{0.00,0.00,0.81}{#1}}
\newcommand{\ConstantTok}[1]{\textcolor[rgb]{0.00,0.00,0.00}{#1}}
\newcommand{\CharTok}[1]{\textcolor[rgb]{0.31,0.60,0.02}{#1}}
\newcommand{\SpecialCharTok}[1]{\textcolor[rgb]{0.00,0.00,0.00}{#1}}
\newcommand{\StringTok}[1]{\textcolor[rgb]{0.31,0.60,0.02}{#1}}
\newcommand{\VerbatimStringTok}[1]{\textcolor[rgb]{0.31,0.60,0.02}{#1}}
\newcommand{\SpecialStringTok}[1]{\textcolor[rgb]{0.31,0.60,0.02}{#1}}
\newcommand{\ImportTok}[1]{#1}
\newcommand{\CommentTok}[1]{\textcolor[rgb]{0.56,0.35,0.01}{\textit{#1}}}
\newcommand{\DocumentationTok}[1]{\textcolor[rgb]{0.56,0.35,0.01}{\textbf{\textit{#1}}}}
\newcommand{\AnnotationTok}[1]{\textcolor[rgb]{0.56,0.35,0.01}{\textbf{\textit{#1}}}}
\newcommand{\CommentVarTok}[1]{\textcolor[rgb]{0.56,0.35,0.01}{\textbf{\textit{#1}}}}
\newcommand{\OtherTok}[1]{\textcolor[rgb]{0.56,0.35,0.01}{#1}}
\newcommand{\FunctionTok}[1]{\textcolor[rgb]{0.00,0.00,0.00}{#1}}
\newcommand{\VariableTok}[1]{\textcolor[rgb]{0.00,0.00,0.00}{#1}}
\newcommand{\ControlFlowTok}[1]{\textcolor[rgb]{0.13,0.29,0.53}{\textbf{#1}}}
\newcommand{\OperatorTok}[1]{\textcolor[rgb]{0.81,0.36,0.00}{\textbf{#1}}}
\newcommand{\BuiltInTok}[1]{#1}
\newcommand{\ExtensionTok}[1]{#1}
\newcommand{\PreprocessorTok}[1]{\textcolor[rgb]{0.56,0.35,0.01}{\textit{#1}}}
\newcommand{\AttributeTok}[1]{\textcolor[rgb]{0.77,0.63,0.00}{#1}}
\newcommand{\RegionMarkerTok}[1]{#1}
\newcommand{\InformationTok}[1]{\textcolor[rgb]{0.56,0.35,0.01}{\textbf{\textit{#1}}}}
\newcommand{\WarningTok}[1]{\textcolor[rgb]{0.56,0.35,0.01}{\textbf{\textit{#1}}}}
\newcommand{\AlertTok}[1]{\textcolor[rgb]{0.94,0.16,0.16}{#1}}
\newcommand{\ErrorTok}[1]{\textcolor[rgb]{0.64,0.00,0.00}{\textbf{#1}}}
\newcommand{\NormalTok}[1]{#1}
\usepackage{graphicx,grffile}
\makeatletter
\def\maxwidth{\ifdim\Gin@nat@width>\linewidth\linewidth\else\Gin@nat@width\fi}
\def\maxheight{\ifdim\Gin@nat@height>\textheight\textheight\else\Gin@nat@height\fi}
\makeatother
% Scale images if necessary, so that they will not overflow the page
% margins by default, and it is still possible to overwrite the defaults
% using explicit options in \includegraphics[width, height, ...]{}
\setkeys{Gin}{width=\maxwidth,height=\maxheight,keepaspectratio}
\IfFileExists{parskip.sty}{%
\usepackage{parskip}
}{% else
\setlength{\parindent}{0pt}
\setlength{\parskip}{6pt plus 2pt minus 1pt}
}
\setlength{\emergencystretch}{3em}  % prevent overfull lines
\providecommand{\tightlist}{%
  \setlength{\itemsep}{0pt}\setlength{\parskip}{0pt}}
\setcounter{secnumdepth}{0}
% Redefines (sub)paragraphs to behave more like sections
\ifx\paragraph\undefined\else
\let\oldparagraph\paragraph
\renewcommand{\paragraph}[1]{\oldparagraph{#1}\mbox{}}
\fi
\ifx\subparagraph\undefined\else
\let\oldsubparagraph\subparagraph
\renewcommand{\subparagraph}[1]{\oldsubparagraph{#1}\mbox{}}
\fi

%%% Use protect on footnotes to avoid problems with footnotes in titles
\let\rmarkdownfootnote\footnote%
\def\footnote{\protect\rmarkdownfootnote}

%%% Change title format to be more compact
\usepackage{titling}

% Create subtitle command for use in maketitle
\newcommand{\subtitle}[1]{
  \posttitle{
    \begin{center}\large#1\end{center}
    }
}

\setlength{\droptitle}{-2em}

  \title{BUDS Training: Error rates, coverage rates, and power Lab}
    \pretitle{\vspace{\droptitle}\centering\huge}
  \posttitle{\par}
  \subtitle{Raven Ico}
  \author{}
    \preauthor{}\postauthor{}
    \date{}
    \predate{}\postdate{}
  

\begin{document}
\maketitle

\section{Problem 1}\label{problem-1}

Each time we make a statistical decision, say based on a confidence
interval or p-value, we are making a decision in the face of
uncertainty, and therefore there is an associated risk of making an
incorrect decision. In hypothesis testing we refer to two types of
errors: type I errors, when we reject a true \(H_0\); and type II
errors, when we fail to reject a false \(H_0\). The type I error rate
can be controlled directly by the construction of our test - in
particular, our \(\alpha\) significance level tells us the maximum
\(P(\mbox{Type I Error})\). The likelihood of a type II error depends on
several things: the sample size, the significance level, the variability
in the data, and the true effect size.

\subsection{Part (a)}\label{part-a}

Suppose we observe Shaq shoot 20 freethrows, and we want to use our
sample to decide if he shoots freethrows with better than 50\% accuracy.
Let \(p\) be his true freethrow percentage, and consider a test that
rejects \(H_0: p \leq 0.5\) when he make \(c\) or more shots out of 20.
\(c/20\) is the critical value for this hypothesis test (i.e.~will
reject the null hypothesis if the sample freethrow percentage is greater
than or equal to \(c/20\)).

Suppose his freethrow percentage is exactly 50\% (so rejecting \(H_0\)
is an error). Make a plot of \(P(\mbox{Type I Error})\) versus \(c\),
for \(c=0,1,\ldots,20\). Which value of \(c\) will give us a test with a
type I error rate closest to but not greater than \(0.05\)?

(hint: notice that \(P(\mbox{Type I Error}) = P(X \geq c)\), where
\(X\sim binomial(20,0.5)\). Be careful to include the \(Pr(X=c)\) in
your calculations).

\subsection{Part (b)}\label{part-b}

Assume now, we don't know the true value of Shaq's free throw
percentage. Use your value of \(c\) from the previous problem and
calculate the value of a type II error, \(Pr(X < c)\), for all values of
freethrow percentage (\(p = X/20\)) where it is possible to make a type
II error (it's not possible to make this error when \(H_0: p \le 0.5\)
is true). Create a plot of type II error vs.~Shaq's true FT\%,

\section{Problem 2}\label{problem-2}

\subsection{Part (a)}\label{part-a-1}

We would like to conduct a clinical trial to compare the mean levels of
expression of a specific protein in patients given drug A versus
patients given drug B. The null hypothesis is that the expression levels
for drug B are less than or equal to those of drug A. Assuming the
protein expression levels in each of the groups is approximately
normally distributed with a pooled standard deviation of 1. The effect
size is the true difference in means between the two groups. Plot the
power of a t-test with significance level 0.05 by the effect size for
sizes \(\{0.05, .1, .15, ... 0.95, 1\}\). Plot different curves for
sample sizes \(n = 50, 100, 150, 200\).

\subsection{Part (b)}\label{part-b-1}

For each sample size, what effect size can you detect with a power of
0.8? Explain this result in words.

\section{Problem 3}\label{problem-3}

\begin{Shaded}
\begin{Highlighting}[]
\KeywordTok{set.seed}\NormalTok{(}\DecValTok{2019}\NormalTok{)}
\NormalTok{n <-}\StringTok{ }\DecValTok{9} \CommentTok{# Sample size}
\NormalTok{alpha <-}\StringTok{ }\DecValTok{1} \CommentTok{# Shape parameter  }
\NormalTok{beta <-}\StringTok{ }\DecValTok{10} \CommentTok{# Scale parameter}

\NormalTok{x <-}\StringTok{ }\KeywordTok{rgamma}\NormalTok{(n, alpha, beta) }\CommentTok{#A random sample to play with}
\NormalTok{ci <-}\StringTok{ }\KeywordTok{t.test}\NormalTok{(x)}\OperatorTok{$}\NormalTok{conf.int[}\DecValTok{1}\OperatorTok{:}\DecValTok{2}\NormalTok{] }\CommentTok{#Get the confidence interval bounds}
\end{Highlighting}
\end{Shaded}

\subsection{Part (a)}\label{part-a-2}

The above piece of code simulates a sample from a gamma distribution and
calcultes the confidence interval for the mean. Simulate 100 random
samples and store the confidence interval and mean for each sample,
arrange the dataframe by the value of the means.

\subsection{Part (b)}\label{part-b-2}

Create a plot of the 100 simulated confidence intervals with the true
mean indicated by a line on the graph.

\subsection{Part (c)}\label{part-c}

Report the percentage of confidence intervals that successfully capture
the true expected value (the coverage rate), and conversely, the
percentage of confidence intervals that don't contain the true value
(the error rate).

\subsection{Part (d)}\label{part-d}

Is the error rate close to what it should be?

\subsection{Part (e)}\label{part-e}

Now run the simulation 1,000 times. Calculate the coverage rate and
error rate and comment. Can you think of any reasons these intervals
might have a higher error rate than they should?


\end{document}
